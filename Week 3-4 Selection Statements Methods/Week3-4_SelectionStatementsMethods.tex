\documentclass[12pt]{article}
\usepackage[margin=0.7in]{geometry}
\usepackage{amsmath}
\usepackage{enumerate}
\usepackage{enumitem}
\usepackage[pdftex]{graphicx}
\usepackage{hyperref}
\usepackage{framed}
\usepackage{listings}
\usepackage{makecell}
\usepackage{pxfonts}
\pagestyle{plain}

\lstset{%
  language=Java,
  basicstyle=\fontsize{9}{11}\selectfont\ttfamily,
  keywordstyle=\bfseries,
  breaklines=true,
  commentstyle = \fontsize{9}{11}\selectfont\ttfamily,
  columns=fullflexible,
  frame=single,
  showstringspaces=false,
  tabsize=4,
  morekeywords = {main},
}

\begin{document}
\begin{center}
	\textbf{CS-200: Programming I}\\
	\textbf{Fall 2017}\\
	\textbf{Northeastern Illinois University}\\
	\textbf{PLTL: Week of 09/18/17}\\
	\textbf{Selection Statements/Methods}
\end{center}

\noindent\textbf{Practice Problem \#1}
\begin{itemize}
	\item Write a program that has the class name \texttt{Problem1} and that has the \texttt{main} method.
	\item Write a second method named \texttt{SumProduct} that takes two integer parameters, \texttt{a} and \texttt{b} and return a String.
	\item The method should calculate the sum and the product  of \texttt{a} and \texttt{b}. If sum is greater than product, program should return \texttt{Sum}. Similarly, if the product is greater than sum, it should return \texttt{Product}. And if the Sum and Product are same, then return \texttt{Tie}. In all the cases, the program should display \texttt{Awesome} if the sum and product are both divisible by 8.
	\item Several sample method calls are provided for you below. You should test your method inside the \texttt{main} method.
\end{itemize}
\begin{center}
\begin{tabular}{| c | c |}
\hline\rule{0pt}{4ex}
Sample Method Usage & return \\
\hline\rule{0pt}{4ex}
\texttt{SumProduct(5, 1)} & \texttt{Sum}\\
\hline\rule{0pt}{4ex}
\texttt{SumProduct(64, 32)} & \texttt{Awesome}\\
\hline\rule{0pt}{4ex}
\texttt{SumProduct(8, 7)} & \texttt{Product}\\
\hline\rule{0pt}{4ex}
\texttt{SumProduct(2, 2)} & \texttt{Tie}\\
\hline
\end{tabular}
\end{center}

\vspace*{0.5cm}
\noindent\textbf{Practice Problem \#2}
\begin{itemize}
	\item Write a program that has the class name \texttt{Problem2} and that has the \texttt{main} method.
	\item Write a second method named \texttt{GetRandomNumber} that takes no parameters, and returns a random integer between 0 to 100.
	\item Write a second method called \texttt{WeatherCheck} that accepts a String \texttt{season} and  an integer \texttt{temp} and returns a String.
	\item You should pass the value returned by \texttt{GetRandomNumber} as an integer parameter when you are calling \texttt{WeatherCheck}.
	\item Check if the season is Fall and if:\\
	temp is between 0-30 return \texttt{It's cold for a fall.}\\
	temp is between 30-60 return \texttt{It's a normal temperature for fall.}\\
	temp is more than 60 return \texttt{It's hot. I cannot believe it is fall.}

	\item Check if the season is Winter and if:\\
	temp is between 0-30 return \texttt{It's normal temperature for winter.}\\
	temp is between 30-60 return \texttt{It's warm. Very nice weather.}\\
	temp is more than 60 return \texttt{It's super-hot for winter. Do you even live in Chicago}?

	\item Check if the season is Spring and if:\\
	temp is between 0-30 return \texttt{It's cold for a spring}.\\
	temp is between 30-60 return \texttt{It's a normal temperature for spring}.\\
	temp is more than 60 return \texttt{It's hot. It appears that summer is already here.}

	\item Check if the season is Summer and if:\\
	temp is between 0-30 return \texttt{It's too cold for summer. Well well, here is Chicago for you}.\\
	temp is between 30-60 return \texttt{It's cold. Is it summer yet}?\\
	temp is more than 60 return \texttt{It's so nice. I love summer}.
	\item Several sample runs are provided for you below. Your output must be formatted \textbf{exactly} like the sample runs below. Note that while your output must be formatted as below, you will not get the same results as this uses random numbers.

\end{itemize}
\begin{center}
\begin{tabular}{| c | c |}
\hline\rule{0pt}{4ex}
Sample Method Usage & return \\
\hline\rule{0pt}{4ex}
\texttt{WeatherCheck(Winter, 72)} & \texttt{It's super-hot for winter. Do you even live in Chicago?}\\
\hline\rule{0pt}{4ex}
\texttt{WeatherCheck(Spring, 35)} & \texttt{It's a normal temperature for spring.}\\
\hline\rule{0pt}{4ex}
\texttt{WeatherCheck(Fall, 68)} & \texttt{It's hot. I cannot believe it is fall.}\\
\hline\rule{0pt}{4ex}
\texttt{WeatherCheck(Summer, 59)} & \texttt{It's cold. Is it summer yet?}\\
\hline
\end{tabular}
\end{center}

\end{document}