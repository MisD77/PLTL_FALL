\documentclass[12pt]{article}
\usepackage[margin=0.8in]{geometry}
\usepackage{amsmath}
\usepackage{enumerate}
\usepackage{enumitem}
\usepackage[pdftex]{graphicx}
\usepackage{hyperref}
\usepackage{framed}
\usepackage{upquote}
\usepackage{listings}
\usepackage{makecell}
\usepackage{pxfonts}
\usepackage[T1]{fontenc}
\pagestyle{plain}

\lstset{%
  language=Java,
  basicstyle=\fontsize{9}{11}\selectfont\ttfamily,
  keywordstyle=\bfseries,
  breaklines=true,
  commentstyle = \fontsize{9}{11}\selectfont\ttfamily,
  columns=fullflexible,
  frame=single,
  showstringspaces=false,
  tabsize=4,
  morekeywords = {main},
}

\begin{document}
\begin{center}
	\textbf{CS-200: Programming I}\\
	\textbf{Fall 2017}\\
	\textbf{Northeastern Illinois University}\\
	\textbf{PLTL: Week of 10/24/17}\\
	\textbf{Nested loops/Arrays}
\end{center}


\noindent\underline{\textbf{Problem \#1}}
\begin{itemize}
	\item Write a program that has the class name \texttt{Problem1} and that has the \texttt{main} method. 
	\item Write a program that asks user to enter the length of a square greater than 1.
	\item Prompt the user to enter the length until they enter number greater than 1.
	\item The program should create a full Square using length.
	\item Your output must match the sample output format \texttt{exactly}.

\end{itemize}
\begin{center}
\begin{minipage}{8cm}

\begin{lstlisting}[escapechar=$]
Enter a square length greater than 1 ( > 1): $\textbf{5}$
$\textbf{5 * * * 1}$
$\textbf{* 4 * 2 *}$
$\textbf{* * 3 * *}$
$\textbf{* 4 * 2 *}$
$\textbf{5 * * * 1}$
\end{lstlisting}
\end{minipage}
\hspace*{.5cm}
\begin{minipage}{8cm}
\begin{lstlisting}[escapechar=$]
Enter a square length greater than 1 ( > 1): $\textbf{-1}$
Enter a square length greater than 1 ( > 1): $\textbf{1}$
Enter a square length greater than 1 ( > 1): $\textbf{3}$
$\textbf{3 * 1}$
$\textbf{* 2 *}$
$\textbf{3 * 1}$
\end{lstlisting}
\end{minipage}
\end{center}
\vspace*{0.5cm}
\noindent\underline{\textbf{Problem \#2}}
\begin{itemize}
	\item Write a program that has the class name \texttt{Problem2} and that has the \texttt{main} method. Leave the \texttt{main} method empty for now.
	\item Write a method named \texttt{closestPower} that takes two parameter, an integer \texttt{n} and an integer array \texttt{a} and returns an integer.
	\item The method should find \texttt{num} which is the value in array \texttt{a} that appears the most. If none of the value is repeated then consider the largest one as \texttt{num}.
	\item Now that you have \texttt{num} and \texttt{n}, find the integer \texttt{i} which is the power of \texttt{n}, such that \texttt{n$^{i}$} is closest to \texttt{num} and return the value of \texttt{i}.
	\item In case of tie, return the smaller value.
	\item Several sample usages are provided for you below. Use the sample usages in the \texttt{main} method to test your code.
\end{itemize}
\begin{center}
\small
\begin{tabular}{| l | c |}
\hline\rule{0pt}{4ex}
Sample Method Usage & Return Value\\
\hline\rule{0pt}{5ex}
\makecell[l]{\texttt{\textbf{int} n1 = 3;}\\ \texttt{\textbf{int}[] a1 = \{ 6, 81, 17, 12, 25, 24, 12 \}; } \\ \texttt{\textbf{int} x1 = closestPower(n1, a1);}} & \texttt{ 2 } \\
\hline\rule{0pt}{5ex}
\makecell[l]{\texttt{\textbf{int} n2 = 4;}\\ \texttt{\textbf{int}[] a2 = \{3, 4, 5, 1, 12 , 67, 3, 1, 1 \}; } \\ \texttt{\textbf{int} x2 = closestPower(n2, a2);}} & \texttt{ 0 } \\
\hline\rule{0pt}{5ex}
\makecell[l]{\texttt{\textbf{int} n3 = 7;}\\ \texttt{\textbf{int}[] a3 = \{ 77, 22, 185, 20, 269, 88 \}; } \\ \texttt{\textbf{int} x3 = closestPower(n3, a3);}} & \texttt{ 3 } \\
\hline\rule{0pt}{5ex}
\makecell[l]{\texttt{\textbf{int} n4 = 2;}\\ \texttt{\textbf{int}[] a4 = \{1, 4, 24, 3, 12, 8 \}; } \\ \texttt{\textbf{int} x2 = closestPower(n4, a4);}} & \texttt{ 4 } \\
\hline
\end{tabular}
\end{center}

\end{document}