\documentclass[12pt]{article}
\usepackage[margin=0.8in]{geometry}
\usepackage{amsmath}
\usepackage{enumerate}
\usepackage{enumitem}
\usepackage[pdftex]{graphicx}
\usepackage{hyperref}
\usepackage{framed}
\usepackage{listings}
\usepackage{makecell}
\usepackage{pxfonts}
\pagestyle{plain}

\lstset{%
  language=Java,
  basicstyle=\fontsize{9}{11}\selectfont\ttfamily,
  keywordstyle=\bfseries,
  breaklines=true,
  commentstyle = \fontsize{9}{11}\selectfont\ttfamily,
  columns=fullflexible,
  frame=single,
  showstringspaces=false,
  tabsize=4,
  morekeywords = {main},
}

\begin{document}
\begin{center}
	\textbf{CS-200: Programming I}\\
	\textbf{Fall 2017}\\
	\textbf{Northeastern Illinois University}\\
	\textbf{PLTL: Week of 10/03/17}\\
	\textbf{Looping/Methods}
\end{center}

\noindent\textbf{Problem \#1}
\begin{itemize}
	\item Write a program that has the class name \texttt{Problem1} and that has the \texttt{main} method.
	\item Prompt the user to enter 10 integers and print out the largest difference and smallest difference obtained by substracting an integer from the one following it.
	\item Depending upon the largest and smallest difference obtained, print out the message accordingly.
	\item Several sample runs are provided for you below. Your output must be formatted \textbf{exactly} like the sample runs below. 
\end{itemize}
\begin{center}
\begin{minipage}{4cm}
\begin{lstlisting}[escapechar=@]
Enter an integer: @\textbf{2}@
Enter an integer: @\textbf{8}@
Enter an integer: @\textbf{17}@
Enter an integer: @\textbf{25}@
Enter an integer: @\textbf{90}@
Enter an integer: @\textbf{45}@
Enter an integer: @\textbf{50}@
Enter an integer: @\textbf{32}@
Enter an integer: @\textbf{09}@
Enter an integer: @\textbf{64}@
Largest difference: 65
Smallest difference: -45 
\end{lstlisting}
\end{minipage}
\hspace*{0.5cm}
\begin{minipage}{4cm}
\begin{lstlisting}[escapechar=@]
Enter an integer: @\textbf{-5}@
Enter an integer: @\textbf{8}@
Enter an integer: @\textbf{-37}@
Enter an integer: @\textbf{64}@
Enter an integer: @\textbf{2}@
Enter an integer: @\textbf{-21}@
Enter an integer: @\textbf{0}@
Enter an integer: @\textbf{99}@
Enter an integer: @\textbf{41}@
Enter an integer: @\textbf{-9}@
Largest difference: 101
Smallest difference: -62

\end{lstlisting}
\end{minipage}
\hspace*{0.5cm}
\begin{minipage}{4cm}
\begin{lstlisting}[escapechar=@]
Enter an integer: @\textbf{77}@
Enter an integer: @\textbf{51}@
Enter an integer: @\textbf{34}@
Enter an integer: @\textbf{65}@
Enter an integer: @\textbf{89}@
Enter an integer: @\textbf{56}@
Enter an integer: @\textbf{-33}@
Enter an integer: @\textbf{-6}@
Enter an integer: @\textbf{7}@
Enter an integer: @\textbf{27}@
Largest difference: 31
Smallest difference: -89

\end{lstlisting}
\end{minipage}\\
\end{center}
\vspace*{0.5cm}
\noindent\textbf{Problem \#2}
\begin{itemize}
	\item Write a program that has the class name \texttt{Problem2} and that has the \texttt{main} method. Leave the \texttt{main} method empty for now.
	\item Write a method named \texttt{allDigitsOdd} that takes one parameter, an integer \texttt{n} and returns a \texttt{boolean}.
	\item Determine whether every digit of the parameter \texttt{n} is odd. Your method should return \texttt{true} if the number consists entirely of odd digits and \texttt{false} if any of its digits are even. \texttt{0, 2, 4, 6 and 8} are even digits, and \texttt{1, 3, 5, 7, 9} are odd digits. 
	\item Hint: You can pull apart a number using \texttt{$\setminus$}10 and \texttt{\%} 10.
	\item Several sample usages are provided for you below. Use the sample usages in the \texttt{main} method to test your code.
\end{itemize}
\begin{center}
\begin{tabular}{| c | c |}
\hline\rule{0pt}{4ex}
Sample Method Usage & Return Value \\
\hline\rule{0pt}{4ex}
\texttt{allDigitsOdd(73925)} & \texttt{false}\\
\hline\rule{0pt}{4ex}
\texttt{allDigitsOdd(59175)} & \texttt{true}\\
\hline\rule{0pt}{4ex}
\texttt{allDigitsOdd(530)} & \texttt{false}\\
\hline\rule{0pt}{4ex}
\texttt{allDigitsOdd(31)} & \texttt{true}\\
\hline
\end{tabular}
\end{center}

\end{document}