
\documentclass[12pt]{article}
\usepackage[margin=0.8in]{geometry}
\usepackage{amsmath}
\usepackage{enumerate}
\usepackage{enumitem}
\usepackage[pdftex]{graphicx}
\usepackage{hyperref}
\usepackage{framed}
\usepackage{upquote}
\usepackage{listings}
\usepackage{makecell}
\usepackage{pxfonts}
\usepackage[T1]{fontenc}
\pagestyle{plain}

\lstset{%
	language=Java,
	basicstyle=\fontsize{9}{11}\selectfont\ttfamily,
	keywordstyle=\bfseries,
	breaklines=true,
	commentstyle = \fontsize{9}{11}\selectfont\ttfamily,
	columns=fullflexible,
	frame=single,
	showstringspaces=false,
	tabsize=4,
	morekeywords = {main},
}

\begin{document}
	\begin{center}
		\textbf{CS-200: Programming I}\\
		\textbf{Fall 2017}\\
		\textbf{Northeastern Illinois University}\\
		\textbf{PLTL: Week of 02/27/17}\\
		\textbf{Arrays/Loops}
	\end{center}
	
	\noindent\textbf{Problem \#1}
	\begin{itemize}
		\item Write a program that has the class name \texttt{Problem1} and that has the \texttt{main} method. 
		\item Write a method named \texttt{multipleOfIndices} that takes one parameter, a positive integer array \texttt{arr} and returns a boolean array. 
		\item For every integer in the integer array, the program should check if the integer is a multiple of the index it is in, and assign the boolean as an element for boolean array at that index. Note that  if the index is \texttt{0$^{th}$} and \texttt{1$^{st}$}, then return \texttt{true} if the remainder is equal to the index when divided by 10 else return \texttt{false}.
		
		\item As a reminder, a number \texttt{m} is a multiple of \texttt{n} if \texttt{m} can be evenly divided into \texttt{n}. For example, \texttt{24} can be divided into \texttt{3} evenly, therefore \texttt{24} is a multiple of \texttt{3}, so the element would get a value of \texttt{true} if \texttt{24} is in a \texttt{3$^{rd}$} index.
		
		\item Create a \texttt{printArray} method that takes a boolean array as a parameter and prints out the elements of the array on the same line separated by a space.
		
		\item Several sample runs are provided for you below. Your output must be formatted \textbf{exactly} like the sample runs below. Use the sample usages in the \texttt{main} method to test your code 
	\end{itemize}
	\begin{center}
		\small
		\begin{tabular}{| c | c |}
			\hline\rule{0pt}{4ex}
			Sample Method Usage & Return Value \\
			\hline\rule{0pt}{5ex}
			\makecell[l]{\texttt{\textbf{int}[] a1 = \{ 1, 21, 5, 9, 12, -50, 47 \};} \\ \texttt{\textbf{boolean}[] b1 = multipleOfIndices(a1)};} & \texttt{\{ false, true, false, true, true, true, false \}}\\
			\hline\rule{0pt}{5ex}
			\makecell[l]{\texttt{\textbf{int}[] a2 = \{5, 3, 77, 34, 43\};} \\ \texttt{\textbf{boolean}[] b2 = multipleOfIndices(a2)};} & \texttt{\{ false, false, false, false, false\}}\\
			\hline\rule{0pt}{5ex}
			\makecell[l]{\texttt{\textbf{int}[] a3 = \{ 30, 22, 42, 8, 15, 27, 6 \};} \\ \texttt{\textbf{boolean}[] b3 = multipleOfIndices(a3)};} & \texttt{\{ true, false, true, false, false, false, true \}}\\
			\hline\rule{0pt}{5ex}
			\makecell[l]{\texttt{\textbf{int}[] a4 = \{ 10, 51, 34, 69, 44, 95\};} \\ \texttt{\textbf{boolean}[] b4 = multipleOfIndices(a4)};} & \texttt{\{ true, true, true, true, true, true \}}\\
			\hline
		\end{tabular}
	\end{center}
	\vspace*{0.5cm}
	\noindent\textbf{Problem \#2}
	\begin{itemize}
		\item Write a program that has the class name \texttt{Problem2} and that has the \texttt{main} method. Leave the \texttt{main} method empty for now.
		\item Write a method named \texttt{greaterThanSum} that takes one parameter, an integer array \texttt{a} and returns a new integer array \texttt{x}.
		\item  The method finds all the terms of an array \texttt{a} that are greater than the sum of all previous terms of the sequence. If there is no such elements, you can return an empty array.
		\item Several sample usages are provided for you below. Use the sample usages in the \texttt{main} method to test your code.Create a \texttt{printArray} method that takes an integer array as a parameter and prints out the elements of the array on the same line separated by a comma and space.
	\end{itemize}
\begin{center}
\small
\begin{tabular}{| l | c |}
	\hline\rule{0pt}{4ex}
	Sample Method Usage & Return Value\\
	\hline\rule{0pt}{5ex}
	\makecell[l]{\texttt{\textbf{int}[] a1 = \{ 1, 4, 16, -19, -12, 2, 5 \}; } \\ \texttt{\textbf{int}[] x1 = greaterThanSum(a1);}} & \texttt{\{1, 4, 16, 2, 5  \}} \\
	\hline\rule{0pt}{5ex}
	\makecell[l]{\texttt{\textbf{int}[] a2 = \{ -1, -2, -4, -12 \}; } \\ \texttt{\textbf{int}[] x2 = greaterThanSum(a2);}} & \texttt{\{ \}} \\
	\hline\rule{0pt}{5ex}
	\makecell[l]{\texttt{\textbf{int}[] a3 = \{  29, -10, 22, 5, -15, 19, 62\}; } \\ \texttt{\textbf{int}[] x3 = greaterThanSum(a3);}} & \texttt{\{29, 22, 62  \}} \\
	\hline\rule{0pt}{5ex}
	\makecell[l]{\texttt{\textbf{int}[] a4 = \{ 5, 8, 17, 50\}; } \\ \texttt{\textbf{int}[] x4 = greaterThanSum(a4);}} & \texttt{\{5, 8, 17, 50 \}} \\
	\hline
		\end{tabular}
	\end{center}
	
\end{document}