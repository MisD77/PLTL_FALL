\documentclass[12pt]{article}
\usepackage[margin=0.8in]{geometry}
\usepackage{amsmath}
\usepackage{enumerate}
\usepackage{enumitem}
\usepackage[pdftex]{graphicx}
\usepackage{hyperref}
\usepackage{framed}
\usepackage{upquote}
\usepackage{listings}
\usepackage{makecell}
\usepackage{pxfonts}
\usepackage[T1]{fontenc}
\pagestyle{plain}

\lstset{%
  language=Java,
  basicstyle=\fontsize{9}{11}\selectfont\ttfamily,
  keywordstyle=\bfseries,
  breaklines=true,
  commentstyle = \fontsize{9}{11}\selectfont\ttfamily,
  columns=fullflexible,
  frame=single,
  showstringspaces=false,
  tabsize=4,
  morekeywords = {main},
}

\begin{document}
\begin{center}
	\textbf{CS-200: Programming I}\\
	\textbf{Fall 2017}\\
	\textbf{Northeastern Illinois University}\\
	\textbf{PLTL: Week of 02/27/17}\\
	\textbf{Arrays/Loops}
\end{center}

\noindent\textbf{Problem \#1}
\begin{itemize}
	\item Write a program that has the class name \texttt{Problem1} and that has the \texttt{main} method.
	\item Prompt the user to enter a positive integer \texttt{n} greater than 1. Assume that the user will enter the correct value. 
	\item Then prompt the user to enter \texttt{n} integers.
	\item Move all the 0's to the end of an array. Maintain the relative position of the other (non-zeros) array elements and print out the new array. 
	\item You should also print out the number of \texttt{0}'s as well as the number of integers that are before \texttt{0}. If there are no \texttt{0}'s then print \texttt{"There are no zeros in the array."}
	\item Several sample runs are provided for you below. Your output must be formatted \textbf{exactly} like the sample runs below. Use the sample usages in the \texttt{main} method to test your code 
\end{itemize}
\begin{center}
\begin{minipage}{12cm}
	\begin{lstlisting}[escapechar=$]
	Enter n ( > 1): $\textbf{8}$
	Enter 8 integers: $\textbf{ 0  0 3 0 2 7 0 9 }$
	The new array with 0's at last: $\textbf{ 3 2 7 9 0 0 0 0 }$
	There are 4 zeros in the array and there are 4 integers before them.
	\end{lstlisting}
\end{minipage}\\
\begin{minipage}{12cm}
	\begin{lstlisting}[escapechar=$]
	Enter n ( > 1): $\textbf{7}$
	Enter 7 integers: $\textbf{4 0 3 0 0 5 8}$
	The new array with 0's at last: $\textbf{ 4 3 5 8 0 0 0 }$
	There are 3 zeros in the array and there are 4 integers before them. 
	\end{lstlisting}
\end{minipage}\\
\begin{minipage}{6cm}
	\begin{lstlisting}[escapechar=$]
	Enter n ( > 1): $\textbf{4}$
	Enter 4 integers: $\textbf{22 6 89 4}$
	There are no zeros in the array.
	\end{lstlisting}
\end{minipage}

\end{center}
\vspace*{0.5cm}
\noindent\textbf{Problem \#2}
\begin{itemize}
	\item Write a program that has the class name \texttt{Problem2} and that has the \texttt{main} method. Leave the \texttt{main} method empty for now.
	\item Write a method named \texttt{palindromeChar} that takes one parameter, a character array \texttt{a} and returns a boolean.
	\item The program checks weather or not the given lists if char array is a palindrome or not. if the characters reads same forward and backwards the method returns \texttt{true} and if not the method returns \texttt{false}.
	\item A palindrome is a sequence of words, numbers or any characters that is the same when written forward or backwards.
	\item For example, an array of \textquotesingle r\textquotesingle , \textquotesingle a\textquotesingle, \textquotesingle d\textquotesingle, \textquotesingle a\textquotesingle, \textquotesingle r\textquotesingle, reads same from forward and backward so it returns \texttt{true} and prints out \texttt{It is a palindrome}. Similarly, \textquotesingle w\textquotesingle, \textquotesingle a\textquotesingle, \textquotesingle t\textquotesingle, \textquotesingle e\textquotesingle, \textquotesingle r\textquotesingle, is not a palindrome and thus return \texttt{false} and prints \texttt{It is not a palindrome}.
	\item Several sample usages are provided for you below. Use the sample usages in the \texttt{main} method to test your code.
\end{itemize}
\begin{center}
\small
\begin{tabular}{| c | c |}
\hline\rule{0pt}{4ex}
Sample Method Usage & Return Value \\
\hline\rule{0pt}{5ex}
\makecell[l]{\texttt{\textbf{char}[] a = \{ \textquotesingle r\textquotesingle,\textquotesingle a\textquotesingle,\textquotesingle c\textquotesingle,\textquotesingle e\textquotesingle,\textquotesingle c\textquotesingle,\textquotesingle a\textquotesingle,\textquotesingle r\textquotesingle \};} \\ \texttt{\textbf{boolean} a1 = palindromeChar(a)};} & \texttt{true}\\
\hline\rule{0pt}{5ex}
\makecell[l]{\texttt{\textbf{char}[] b = \{ \textquotesingle w\textquotesingle,\textquotesingle a\textquotesingle,\textquotesingle t\textquotesingle,\textquotesingle e\textquotesingle,\textquotesingle r\textquotesingle \};} \\ \texttt{\textbf{boolean} b1 = palindromeChar(b)};} & \texttt{false}\\
\hline\rule{0pt}{5ex}
\makecell[l]{\texttt{\textbf{char}[] c = \{ \textquotesingle f\textquotesingle,\textquotesingle o\textquotesingle,\textquotesingle o\textquotesingle,\textquotesingle f\textquotesingle,\textquotesingle a\textquotesingle,\textquotesingle a\textquotesingle,\textquotesingle r\textquotesingle \};} \\ \texttt{\textbf{boolean} c1 = palindromeChar(c)};} & \texttt{false}\\
\hline\rule{0pt}{5ex}
\makecell[l]{\texttt{\textbf{char}[] d = \{ \textquotesingle c\textquotesingle,\textquotesingle b\textquotesingle,\textquotesingle a\textquotesingle,\textquotesingle a\textquotesingle,\textquotesingle b\textquotesingle,\textquotesingle c\textquotesingle \};} \\ \texttt{\textbf{boolean} d1 = palindromeChar(d)};} & \texttt{true}\\
\hline
\end{tabular}
\end{center}

\end{document}