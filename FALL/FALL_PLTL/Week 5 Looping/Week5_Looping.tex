\documentclass[12pt]{article}
\usepackage[margin=0.8in]{geometry}
\usepackage{amsmath}
\usepackage{enumerate}
\usepackage{enumitem}
\usepackage[pdftex]{graphicx}
\usepackage{hyperref}
\usepackage{framed}
\usepackage{listings}
\usepackage{makecell}
\usepackage{pxfonts}
\pagestyle{plain}

\lstset{%
  language=Java,
  basicstyle=\fontsize{9}{11}\selectfont\ttfamily,
  keywordstyle=\bfseries,
  breaklines=true,
  commentstyle = \fontsize{9}{11}\selectfont\ttfamily,
  columns=fullflexible,
  frame=single,
  showstringspaces=false,
  tabsize=4,
  morekeywords = {main},
}

\begin{document}
\begin{center}
	\textbf{CS-200: Programming I}\\
	\textbf{Fall 2017}\\
	\textbf{Northeastern Illinois University}\\
	\textbf{PLTL: Week of 02/13/17}\\
	\textbf{Looping}
\end{center}


\noindent\textbf{Practice Problem \#1}
\begin{itemize}
	\item Write a program that has the class name \texttt{Problem1} and that has the \texttt{main} method and two method called \texttt{SumFromProduct} and \texttt{SumFactorial}.
	\item In the main method prompt the user to enter positive  number up to 20(inclusive)for up to 5 times. It then counts and adds the positive values. 
	\item Your prompt terminates whenever the user enters 5 number or non-positive number. You can assume that there is atleast one non zero number.
	\item Print out the count of the positive numbers, their product.
	\item Several sample runs are provided for you below. Your output must be formatted \textbf{exactly} like the sample runs below.
\end{itemize}
\begin{center}
\begin{minipage}{7cm}
\begin{lstlisting}[escapechar=@]
Enter a positive integer: @\textbf{12}@
Enter a positive integer: @\textbf{17}@
Enter a positive integer: @\textbf{2}@
Enter a positive integer: @\textbf{0}@
Number of positive numbers is: 3
The product of positive numbers is: 408
\end{lstlisting}
\end{minipage}
\hspace*{0.5cm}
\begin{minipage}{7cm}
\begin{lstlisting}[escapechar=@]
Enter a positive integer: @\textbf{6}@
Enter a positive integer: @\textbf{5}@
Enter a positive integer: @\textbf{7}@
Enter a positive integer: @\textbf{3}@
Enter a positive integer: @\textbf{19}@
Number of positive numbers is: 5
The product of positive numbers is: 11970
\end{lstlisting}
\end{minipage}
\begin{minipage}{7cm}
\begin{lstlisting}[escapechar=@]
Enter a positive integer: @\textbf{356}@
Enter a positive integer: @\textbf{-6}@
Number of positive numbers is: 1
The product of positive numbers is: 356
\end{lstlisting}
\end{minipage}
\end{center}
\noindent\textbf{Practice Problem \#2}
\begin{itemize}
	\item This is a Follow up from \texttt{Problem 1}. Write a method named \texttt{SumFromProduct} that takes an integer and returns an integer.
	\item The method parameter is a product that you get from \texttt{Problem 1}. The method then calculates the sum of the digits in the number and returns the value.
	\item If the sum is greater than 10(exclusive) then reduce the sum again to value between 1 and 10(inclusive) by adding the individual digits.
	\item Several sample runs are provided for you below. Your output must be formatted \textbf{exactly} like the sample runs below.
\end{itemize}
\begin{center}
\begin{tabular}{| c | c |}
\hline\rule{0pt}{4ex}
Sample Method Usage & return \\
\hline\rule{0pt}{4ex}
\texttt{SumFromProduct(408)} & \texttt{3}\\
\hline\rule{0pt}{4ex}
\texttt{SumFromProduct(11970)} & \texttt{9}\\
\hline\rule{0pt}{4ex}
\texttt{SumFromProduct(356)} & \texttt{5}\\
\hline
\end{tabular}
\end{center}
	

\noindent\textbf{Practice Problem \#3}
\begin{itemize}
	\item This is a follow up from \texttt{Problem 2}. Write a method named \texttt{SumFactorial} that takes an integer and returns an integer.
	\item The input value is the sum that you get from \texttt{Problem 2}. The method then calculates and returns the factorial of the sum.
	\item The factorial of a number \texttt{n} (called \texttt{n} factorial or \texttt{n!}) is the product of the integers from \texttt{1} up to and including \texttt{n}.
	\item Several sample runs are provided for you below. Your output must be formatted \textbf{exactly} like the sample runs below.
\end{itemize}
\begin{center}
\begin{tabular}{| c | c |}
\hline\rule{0pt}{4ex}
Sample Method Usage & return \\
\hline\rule{0pt}{4ex}
\texttt{SumFactorial(3)} & \texttt{6}\\
\hline\rule{0pt}{4ex}
\texttt{SumFactorial(9)} & \texttt{362880}\\
\hline\rule{0pt}{4ex}
\texttt{SumFactorial(5)} & \texttt{120}\\
\hline
\end{tabular}
\end{center}
\noindent You should call \texttt{SumFromProduct} and \texttt{SumFactorial} from the main method in \texttt{Problem1} only. If you implemented your methods correctly, the outpur should match the following:
\begin{center}
\begin{minipage}{7cm}
\begin{lstlisting}[escapechar=@]
Enter a positive integer: @\textbf{12}@
Enter a positive integer: @\textbf{17}@
Enter a positive integer: @\textbf{2}@
Enter a positive integer: @\textbf{0}@
Number of positive numbers is: 3
The product of positive numbers is: 408
The sum of the product is: 3
The factorial of the sum is: 6
\end{lstlisting}
\end{minipage}
\hspace*{0.5cm}
\begin{minipage}{7cm}
\begin{lstlisting}[escapechar=@]
Enter a positive integer: @\textbf{6}@
Enter a positive integer: @\textbf{5}@
Enter a positive integer: @\textbf{7}@
Enter a positive integer: @\textbf{3}@
Enter a positive integer: @\textbf{19}@
Number of positive numbers is: 5
The product of positive numbers is: 11970
The sum of the product is: 9
The factorial of the sum is: 362880
\end{lstlisting}
\end{minipage}
\begin{minipage}{7cm}
\begin{lstlisting}[escapechar=@]
Enter a positive integer: @\textbf{356}@
Enter a positive integer: @\textbf{-6}@
Number of positive numbers is: 1
The product of positive numbers is: 356
The sum of the product is: 5
The factorial of the sum is: 120
\end{lstlisting}
\end{minipage}
\end{center}

\end{document}