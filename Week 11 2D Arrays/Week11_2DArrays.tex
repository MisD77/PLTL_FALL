\documentclass[12pt]{article}
\usepackage[margin=0.8in]{geometry}
\usepackage{amsmath}
\usepackage{enumerate}
\usepackage{enumitem}
\usepackage[pdftex]{graphicx}
\usepackage{hyperref}
\usepackage{framed}
\usepackage{upquote}
\usepackage{listings}
\usepackage{makecell}
\usepackage{pxfonts}
\usepackage[T1]{fontenc}
\pagestyle{plain}

\lstset{%
  language=Java,
  basicstyle=\fontsize{9}{11}\selectfont\ttfamily,
  keywordstyle=\bfseries,
  breaklines=true,
  commentstyle = \fontsize{9}{11}\selectfont\ttfamily,
  columns=fullflexible,
  frame=single,
  showstringspaces=false,
  tabsize=4,
  morekeywords = {main},
}

\begin{document}
\begin{center}
	\textbf{CS-200: Programming I}\\
	\textbf{Fall 2017}\\
	\textbf{Northeastern Illinois University}\\
	\textbf{PLTL: Week of 04/12/17}\\
	\textbf{2D Arrays}
\end{center}


\noindent\underline{\textbf{Problem \#1}}
\begin{itemize}
	\item Write a program that has the class name \texttt{Problem1} and that has the \texttt{main} method. Leave the \texttt{main} method empty for now.
	\item Write a method named \texttt{deepReverse} that takes one parameter, a 2-dimensional (2D) integer array named \texttt{arr} and returns a new 2D integer array.
	\item The method should create a new array \texttt{a} such that rows and columns are the reverse of the array \texttt{arr}, such that first row of the array \texttt{arr} is the last row of the new array, second row is the second last row of the new array and so on.  
	\item Similarly, the first column in the array \texttt{arr} is the last column in the new array, second column is the second last column in the new array and so on. See sample usage below.
	\item Create a \texttt{printArray} method that takes a 2D integer array as a parameter and prints out the elements of each row on its own line separated by spaces.
	\item Several sample usages are provided for you below. Use the sample usages in the \texttt{main} method to test your code (and use the \texttt{printArray} method to print out the results of calling the \texttt{deepReverse} method!).
\end{itemize}

\noindent\begin{center}
\small
\begin{tabular}{| l | l |}
\hline\rule{0pt}{4ex}
Sample Method Usage & Return Value\\
\hline\rule{0pt}{5ex}
\makecell[l]{\texttt{\textbf{int}[][] arr1 = \{ \{ 1, 2 , 4, 0\},}\\ \hspace*{3.3cm}\texttt{ \{ 3, 4, 5, 6 \},}\\ \hspace*{3.3cm}\texttt{ \{ 7, 8, 9, 12 \} \};}\\ \texttt{\textbf{int}[][] a1 = deepReverse(arr1);} } & \makecell[l]{\texttt{\{ \{ 12, 9, 8, 7 \},}\\ \hspace*{0.4cm}\texttt{\{ 6, 5, 4, 3 \},}\\ \hspace*{0.4cm}\texttt{\{ 0, 4, 2, 1 \} \};} }\\
\hline\rule{0pt}{5ex}
\makecell[l]{\texttt{\textbf{int}[][] arr2 = \{ \{ 2, 8 \},}\\ \hspace*{2.6cm}\texttt{ \{ 7, 20 \}, }\\ \hspace*{2.6cm}\texttt{ \{ 9, 3 \}, }\\ \hspace*{2.6cm}\texttt{ \{ 5, 12 \} \}; }\\ \texttt{\textbf{int}[][] a2 = deepReverse(arr2);} } & \makecell[l]{\texttt{\{ \{12, 5 \},}\\ \hspace*{0.4cm}\texttt{\{ 3, 9 \},}\\ \hspace*{0.4cm}\texttt{\{20, 7 \},}\\ \hspace*{0.4cm}\texttt{\{ 8, 2 \} \};} }\\
\hline
\end{tabular}
\end{center}

\vspace*{0.5cm}
\noindent\underline{\textbf{Problem \#2}}
\begin{itemize}
	\item Write a program that has the class name \texttt{Problem2} and that has the \texttt{main} method. Leave the \texttt{main} method empty for now.
	\item Write a method named \texttt{isPrime} that takes one parameter, a 2-dimensional (2D) integer array named \texttt{arr} and returns a new 2D boolean array. 
	\item The method checks for every element in 2D array if it is a prime number or not. If it is a prime, then element in boolean array at that index would be \texttt{true} else \texttt{false}.
	\item Create a \texttt{printArray} method that takes a 2D boolean array as a parameter and prints out the elements of each row on its own line separated by spaces.
	\item Several sample usages are provided for you below. Use the sample usages in the \texttt{main} method to test your code (and use the \texttt{print2DArray} method to print out the results of calling the \texttt{isPrime} method!).
\end{itemize}

\noindent\begin{center}
\small
\begin{tabular}{| c | l |}
	\hline\rule{0pt}{4ex}
	Sample Method Usage & Return Value\\
	\hline\rule{0pt}{5ex}
	\makecell[l]{\texttt{\textbf{int}[][] a = \{ \{ 4, 13, 10, 3, 9 \},}\\ \hspace*{2.8cm}\texttt{\{ 14, 19, 43, 5 \},}\\ \hspace*{2.8cm}\texttt{\{ 31, 17, 40, 11 \} \};}\\ \texttt{\textbf{boolean}[][] a1 = isPrime(a);} } & \makecell[l]{\texttt{\{ \{ false, true, false, true, false \},}\\ \hspace*{0.4cm}\texttt{\{ false, true, true, true \},}\\ \hspace*{0.4cm}\texttt{\{ true, true, false, true \} \};} }\\
	\hline\rule{0pt}{5ex}
	\makecell[l]{\texttt{\textbf{int}[][] b = \{ \{ 89, 7, 9 \},}\\ \hspace*{2.8cm}\texttt{\{ 25, 39 \},}\\ \hspace*{2.8cm}\texttt{\{ 133, 29, 41 \} \};}\\ \texttt{\textbf{boolean}[][] b1 = isPrime(b);} } & \makecell[l]{\texttt{\{ \{ false, true, false \},}\\ \hspace*{0.4cm}\texttt{\{ false, false \},}\\ \hspace*{0.4cm}\texttt{\{true, true, true \} \};} }\\
	\hline
\end{tabular}
\end{center}

\end{document}