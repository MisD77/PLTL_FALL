\documentclass[12pt]{article}
\usepackage[margin=0.7in]{geometry}
\usepackage{amsmath}
\usepackage{enumerate}
\usepackage{enumitem}
\usepackage[pdftex]{graphicx}
\usepackage{hyperref}
\usepackage{framed}
\usepackage{listings}
\usepackage{pxfonts}
\pagestyle{plain}

\lstset{%
  language=Java,
  basicstyle=\fontsize{9}{11}\selectfont\ttfamily,
  keywordstyle=\bfseries,
  breaklines=true,
  commentstyle = \fontsize{9}{11}\selectfont\ttfamily,
  columns=fullflexible,
  frame=single,
  showstringspaces=false,
  tabsize=4,
  morekeywords = {main},
}

\begin{document}
\begin{center}
	\textbf{CS-200: Programming I}\\
	\textbf{Fall 2017}\\
	\textbf{Northeastern Illinois University}\\
	\textbf{PLTL: Week of 09/11/17}\\
	\textbf{Selection Statements/Math}
\end{center}

\noindent\textbf{Practice Problem \#1}
\begin{itemize}
	\item Write a program that has the class name \texttt{IsFactor} and that has the \texttt{main} method.
	\item The program should ask the user to enter two integers greater than 1.
	\item The program should determine whether the smaller of the two integers is a factor of the larger of the two integers (regardless of which order the two numbers are entered in). 
	\item Based on whether either one is a factor of another, print out the appropriate message.
	\item As a reminder, a factor of one number is any whole number that can divide the number into evenly. For example, 8 can divide 64 evenly, therefore 8 is a factor of 64. 
	\item Several sample runs are provided for you below. Format your output to match the sample output.
\end{itemize}
\begin{center}
\begin{minipage}{4.5cm}
\begin{lstlisting}[escapechar=!]
Enter the first number: !\textbf{5}!
Enter the second number: !\textbf{65}!
5 is a factor of 65
\end{lstlisting}
\end{minipage}
\hspace*{0.5cm}
\begin{minipage}{4.5cm}
\begin{lstlisting}[escapechar=!]
Enter the first number: !\textbf{35}!
Enter the second number: !\textbf{2}!
2 is not a factor of 35
\end{lstlisting}
\end{minipage}
\end{center}

\vspace*{0.5cm}
\noindent\textbf{Practice Problem \#2}
\begin{itemize}
	\item Write a program that has the class name \texttt{RightAngledTriangle} and that has the \texttt{main} method.
	\item The program should ask the user to enter three numbers \texttt{a}, \texttt{b} and \texttt{c} and check if they are the sides of right angled triangle.
	\item Using the pythagorean theorem, you can check if it is a right angled triangle or not. If it is a right angled triangle then print \texttt{Right angled triangle} if it is not then print \texttt{Not a right angled triangle}. 
	\item As per the theorem the sum of the square of two sides(legs) is equal to the square of the longest side(hypotenuse). 
	\item Below is the formula given to calculate the hypotenuse and you must use Math.Sqrt and Math.pow to calculate it. check the style with prof.
\begin{center}
	$\rm{\mathtt{\displaystyle}\displaystyle h =  \sqrt{{\displaystyle}\displaystyle b^2 + \displaystyle p^2}}$
\end{center}
	\item Several sample runs are provided below. Format your output to match the sample output.
\end{itemize}
\begin{center}
\begin{minipage}{4cm}
\begin{lstlisting}[escapechar=!]
Enter s1: !\textbf{3}!
Enter s2: !\textbf{5}!
Enter s3: !\textbf{4}!
Right angled triangle
\end{lstlisting}
\end{minipage}
\hspace*{0.5cm}
\begin{minipage}{4.5cm}
\begin{lstlisting}[escapechar=!]
Enter s1: !\textbf{9}!
Enter s2: !\textbf{5}!
Enter s3: !\textbf{12}!
Not a right angled triangle
\end{lstlisting}
\end{minipage} \\
\vspace*{0.1cm}
\begin{minipage}{4cm}
\begin{lstlisting}[escapechar=!]
Enter s1: !\textbf{4.4}!
Enter s2: !\textbf{3.3}!
Enter s3: !\textbf{5.5}!
Right angled triangle
\end{lstlisting}
\end{minipage} 
\end{center}	

\vspace*{0.5cm}
\noindent\textbf{Practice Problem \#3}
\begin{itemize}
	\item Write a program that has the class name \texttt{EvenOrOdd} and that has the \texttt{main} method.
	\item The program should ask the user to enter a \texttt{4} bit binary number (\texttt{1}'s \& \texttt{0}'s) and then convert it into the decimal number.
	\item For example, if the input is \texttt{1101}, the output should be \texttt{13} (Hint: Break the number into digits and then convert each digit to a value for a single digit)
	\item If the digits are \texttt{$n_1$}, \texttt{$n_2$}, \texttt{$n_3$} and \texttt{$n_4$}, the decimal equivalent is \texttt{8$n_1$} + \texttt{4$n_2$} + \texttt{2$n_3$} + \texttt{1$n_4$}.
	\item After you get the decimal value of the \texttt{4} bit number, print out the value and check if the value is even or odd and print accordingly.
	\item Several sample runs are provided for you below. Format your output to match the sample output.
\end{itemize}
\begin{center}
\begin{minipage}{5.25cm}
\begin{lstlisting}[escapechar=!]
Enter a 4-bit integer: !\textbf{0011}!
The decimal value of 0011 is 3
3 is an odd number
\end{lstlisting}
\end{minipage}
\hspace*{0.5cm}
\begin{minipage}{5.25cm}
\begin{lstlisting}[escapechar=!]
Enter a 4-bit integer: !\textbf{1010}!
The decimal value of 1010 is 10
10 is an even number
\end{lstlisting}
\end{minipage}
\hspace*{0.5cm}
\end{center}	

\end{document}